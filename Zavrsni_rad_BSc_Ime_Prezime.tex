% Ovo je glavni fajl za generiranje predloska dokumenta doktorske disertacije Elektrotehnickog fakulteta u Sarajevu. 
% Verzija: 11. maj 2018. godine 

% Autor: Emir Sokic , esokic@etf.unsa.ba
% Bazirano na predlosku koji se koristi na FER Zagreb

%%%%%%%%%%%%%%%%%%%%%%%%%%%%%%%%%%%%%%%%%%%%%%%%%%%%%%%%%%%%%%%%%%%%%%
%%%%%%%%%%%%%%%%%%%%%%%%% POSTAVKE %%%%%%%%%%%%%%%%%%%%%%%%%%%%%%%%%%%
%%%%%%%%%%%%%%%%%%%%%%%%% NE MIJENJATI %%%%%%%%%%%%%%%%%%%%%%%%%%%%%%%
%%%%%%%%%%%%%%%%%%%%%%%%%%%%%%%%%%%%%%%%%%%%%%%%%%%%%%%%%%%%%%%%%%%%%%

\documentclass[12pt,oneside, a4paper]{book}

%Ovo su paketi koji se koriste za kompajliranje dokumenta
%Vecina paketa je ukljucena po defaultu u texlive 2013 i novijim verzijama
\usepackage{etex}
\usepackage{xcolor}
\usepackage[pdftex]{graphicx}
\usepackage{rotating}
\usepackage{epsfig}
\usepackage{epstopdf}
\usepackage[T1]{fontenc}
\usepackage[utf8]{inputenc}
\usepackage{cmap}
\usepackage[croatian]{babel}
\usepackage[unicode]{hyperref}
\usepackage{mathptmx}
\usepackage{amscd}
\usepackage{amssymb}
\usepackage{amsmath}
\usepackage{amsfonts}
\usepackage[left=2.5cm,right=2.5cm,top=2.5cm,bottom=2.5cm]{geometry}
\usepackage{setspace} 
\usepackage{hhline}
\usepackage{enumerate}
\usepackage{delarray}
\usepackage{array}  
\usepackage{tabularx} 
\usepackage{multirow}  
\usepackage[bf, font=small]{caption}
\usepackage[labelfont=small, font=small]{subcaption}
\usepackage{wasysym}
\usepackage{subeqnarray}
\usepackage{pdflscape} % setting page into landscape view
\usepackage{enumitem} % for itemize lists
\usepackage[toc,page]{appendix}
\newcommand{\HRule}{\rule{\linewidth}{0.5mm}}
\usepackage{makeidx}
\usepackage{nomencl}
\usepackage{listings}
\lstset{basicstyle=\ttfamily,breaklines=true}
\usepackage{courier}
% Podesavanje izgleda zaglavlja i podnozja strana
\usepackage{fancyhdr}
\usepackage{float}


\fancypagestyle{plain}{%
  \fancyhf{}% Clear header/footer
  \fancyfoot[OR]{{\thepage}}%
  \fancyfoot[EL]{{\thepage}}%
  \renewcommand{\headrulewidth}{0pt}%
}

% required for printing index
% use \index{name} in text
%\usepackage{makeidx}
%\makeindex
% required for printing nomenclature
% use \nomenclature{symbol}{description} in text
%\usepackage{nomencl}
%\makenomenclature
%\renewcommand{\nomname}{Popis oznaka}


%Opcionalno
%\linespread{1.3}
%\setlist{nolistsep}   % setting for itemize lists
%\renewcommand{\thefootnote}{\fnsymbol{footnote}}  % to get unnumbered footnotes

% Adding a dot after chapter number in TOC 
%\let\savenumberline\numberline
%\def\numberline#1{\savenumberline{#1.}}

%\pagestyle{fancyplain}

%\rfoot{\thepage}
% iskljucivanje broja strane iz Sadrzaja, Popisa slika i Popisa tabela
\AtBeginDocument{\addtocontents{toc}{\protect\thispagestyle{empty}}}
\AtBeginDocument{\addtocontents{lof}{\protect\thispagestyle{empty}}}
\AtBeginDocument{\addtocontents{lot}{\protect\thispagestyle{empty}}}

%\rhead{\slshape \nouppercase \leftmark}
%\lhead{} %delete left header


%Podesavanje izgleda referenci
\usepackage[square, numbers, sort]{natbib} 

%Promjena naziva pojedinih poglavlja sa Hrvatskog na Bosanski
% Bibliography u "Literatura"
\addto\captionscroatian{%
  \renewcommand{\bibname}{Literatura}
  \renewcommand{\tablename}{Tabela}
  \renewcommand{\nomname}{Popis oznaka}
  \renewcommand{\indexname}{Indeks pojmova}
  \renewcommand{\lstlistingname}{Program}
}
%"Popis tablica" u "Popis tabela"
\addto\captionscroatian{\renewcommand{\listtablename}{Popis tabela}}
\addto\captionscroatian{\renewcommand\appendixname{Prilog}}
\addto\captionscroatian{\renewcommand\appendixpagename{Prilozi}}
\renewcommand\appendixtocname{Prilozi}

\makeindex
\makenomenclature

%\usepackage{etoolbox}
%\patchcmd{\chapter}{\thispagestyle{plain}}{\thispagestyle{fancyplain}}{}{}


\begin{document}

%%%%%%%%%%%%%%%%%%%%%%%%%%%%%%%%%%%%%%%%%%%%%%%%%%%%%%%%%%%%%%%%%%%%%%
%%%%%%%%%%%%%%%%%%%%%%%%% OSNOVNI DOKUMENT %%%%%%%%%%%%%%%%%%%%%%%%%%%
%%%%%%%%%%%%%%%%%%%%%%%%%%%%%%%%%%%%%%%%%%%%%%%%%%%%%%%%%%%%%%%%%%%%%%

\frontmatter

%%%%%%%%%%%%%%%%%%%% NASLOVNA STRANA %%%%%%%%%%%%%%%%%%%%%%%%
\begin{titlepage}
\begin{center}

\includegraphics[width=0.25\textwidth]{etf-logo.png}~\\[0.1cm]
\textsc{\Large Univerzitet u Sarajevu}\\[0.2cm]  
\textsc{\Large Elektrotehnički fakultet}\\[0.2cm] 
\textsc{\Large Odsjek za telekomunikacije}\\[3cm]\HRule \\[0.5cm] 
{\huge \bfseries Utjecaj Nelinearnog Preslušavanja (XPM) u WDM Sistemima} \\[0.4cm] 
\HRule \\[0.5cm]

\textsc{\Large Projektni zadatak}\\[0.4cm]
\textsc{\Large - Optički telekomunikacioni sistemi - }\\[5cm]
\end{center}
% Author and supervisor 
% Blok sa studentima lijevo i profesorom desno
    \begin{flushleft}
        \textbf{Studenti:} \\
        Ena Šabanović, 2476/18945 \\
        Amsal Škaljo, 2466/18671 \\
        Bakir Šatara, 2385/19004
    \end{flushleft}

    \vspace{-3cm} % pomjera profesorov blok prema gore

    \begin{flushright}
        \textbf{Profesor:} \\
        Prof. dr. Darijo Raca
    \end{flushright}

    \vfill
\begin{center}
% Bottom of the page  
{\large Sarajevo, \\decembar 2025.}

\end{center} 
\end{titlepage}
%%%%%%%%%%%%%%%%%%%%% SADRŽAJ %%%%%%%%%%%%%%%%%%%%%%%%%
%\clearpage

\tableofcontents

%%%%%%%%%%%%%%% POPIS SLIKA %%%%%%%%%%%%%%%%%%%%%%%%%%%
%\clearpage



\cleardoublepage % start new page

\pagestyle{fancyplain} % puts headers/footers back on
\fancyhf{}
\lhead{\nouppercase{\fancyplain{}{\leftmark}}}
\renewcommand{\chaptermark}[1]{\markboth{#1}{}}
\renewcommand{\footrulewidth}{0.4pt} %draw foot line
\lfoot{\slshape Utjecaj Nelinearnog Preslušavanja (XPM) u WDM Sistemima}
\rfoot{\thepage}
\cfoot{}

%%%%%%%%%%%%%%%%%%%%%%%%%%%%%%%%%%%%%%%%%%%%%%%%%%%%%%%%%%%%%%%%%%%%
\chapter*{Uvod}
\addcontentsline{toc}{chapter}{Uvod}

\mainmatter
\include{poglavlje_1}
\include{poglavlje_2}
\include{poglavlje_3}
\include{zakljucak}
%%%%%%%%%%%%%%%%%%%%%%%%%%%%%%%%%%%%%%%%%%%%%%%%%%%%%%%%%%%%%%%%%%%%
%%%%%%%%%%%%%%%%%%%%%%%%% POGLAVLJA %%%%%%%%%%%%%%%%%%%%%%%%%%%%%%%%
%%%%%%%%%%%%%%%%%%%%%%%%%%%%%%%%%%%%%%%%%%%%%%%%%%%%%%%%%%%%%%%%%%%%

%Poglavlja je najbolje raditi u odvojenim fajlovima
%Poglavlje 1

%%%%%%%%%%%%%%%%%%%%%%%%%%%%%%%%%%%%%%%%%%%%%%%%%%%%%%%%%%%%%%%%%%%

\backmatter
\listoffigures
\addcontentsline{toc}{chapter}{Popis slika}

%%%%%%%%%%%%%%% POPIS TABELA %%%%%%%%%%%%%%%%%%%%%%%%%%
%\clearpage
%\listoftables
%\addcontentsline{toc}{chapter}{Popis tabela}
%%%%%%%%%%%%%%%%%%%%%%% LITERATURA %%%%%%%%%%%%%%%%%%%%%%%%%%%%%%%%%
\addcontentsline{toc}{chapter}{Literatura}
\bibliographystyle{IEEEtranETF} 
\bibliography{literatura}


\end{document}
